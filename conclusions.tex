\chapter{Synthesis}

Together, the OSSEs in chapter \ref{osse_chapter} and real-data experiments in chapter \ref{real_chapter} provide two different means to assess various assimilation strategies and the impact of different datasets on convective-scale analyses and forecasts of developing convective systems. Several common themes emerged between the experiments within Part I and Part II. A synthesis of some of the key results of the dissertation are provided in this chapter.

\paragraph{Storm mergers and upscale growth}
In both the OSSEs and real-data experiments, periods of storm interactions and mergers were not analyzed or forecast well by the EnKF system. In the OSSEs, analysis errors increased during the initial period of mergers, while observation innovations increased during a storm merger within CI2 in the real-data experiments. As a result, improving forecasts of these types of transition events appears to pose a greater challenge compared to events composed of more isolated convection.

\paragraph{Radar data localization}
For both types experiments, increased horizontal localization for radar data improved the results, albiet modestly in the real-data experiments. Thus, the real-data experiments support the OSSE findings in Part I that a larger horizontal localization for radar data, up to 18 km, are not detrimental, and may be beneficial, to EnKF analyses of convection. Results were mixed to the sensitivity of vertical localization. In the OSSEs, a smaller vertical localization reduced analysis error, while this change was detrimental to the 29 May 2012 forecasts, primarily in the prediction of an isolated supercell. It is evident that there are situations when larger vertical localization is beneficial to analyses of convection, such as when the storm is spinning up in the analyses and/or when a storm is not ideally placed with respect to observational assets (e.g. if a storm is very near to the radar, the upper parts of the storm may be more poorly observed). With regards to the vertical localization, the real-data experiments did not support the OSSE findings, and thus use of a vertical localization around 6 km remains an appropriate choice for radar data originating from WSR-88Ds.

\paragraph{Importance of multi-scale data assimilation}
The current real-data experiments underscore the importance of multi-scale DA systems in the production of accurate short-term ensemble forecasts of convection out to three-hours. This has been demonstrated in previous work for isolated convection (e.g. \citealt{stensrudgao10}), and is reinforced the present work for a significantly more complex case. Both the forecasts of CI in section 4.5, and the forecasts of convective evolution in section 4.6, exhibited sensitivity to the mesoscale environment, with improved forecasts as the mesoscale environment was represented more realistically. Adjustments to the mesoscale environment through surface DA were able to improve convective forecasts by adjusting surface features and reducing surface moisture biases, while errors in the cloud-layer wind speed led to storm motions that were slower than observed and led for fairly large anvil-level reflectivity biases. For complex convective evolutions, these types of errors can have non-linear effects on the resulting forecast, but can be ameliorated by assimilating both meso- and convective-scale observations, thus embedding convection within an accurate mesoscale environment.

\paragraph{Using data assimilation to diagnose errors}
DA was used both to create forecast initial conditions and to isolate and estimate the magnitude of errors of various components of the DA system, including model error, forward operator error, instrument errors, and mesoscale environment error. For example, in the surface DA experiments, a positive moisture bias was present in the analyses due to a previously known surface moisture bias. In the experiments that assimilated radar observations, various biases were identified due to model error from parameterization of microphysics, forward operator error, and errors in the mesoscale environment. Knowledge of these errors and their impact on the analyses should guide future efforts to improve the DA system, ideally leading to better forecasts.

In section 4.6.2, errors were identified primarily through the use of innovation statistics. Advanced methods to utilize the innovation diagnostics on-line, during the DA process, have been proposed and their applications to convective-scale EnKF are worthy of future work \citep{deedasilva98}. The adaptive inflation algorithm employed within DART provides another mechanism for identifying systematic errors in the analyses. This algorithm produces inflation fields for each component of the model state. The algorithm will increase prior-state inflation values where observations routinely fall outside of the prior ensemble, inflating the prior spread before DA. Spread deficiencies can occur due to inhomogeneities in the observing network, but also if analyses or observations are biased, producing inappropriate magnitudes of ensemble variance \citep{anderson09}.

To illustrate this, the domain-maximum prior inflation values (posterior inflation is not employed in any of the present experiments) were computed in SFCRAD5H for 8 state fields (U10, V10, T2, Q2, QRAIN, QSNOW, QGRAUP, and REFL). At the surface, Q2 has a larger value of domain-maximum prior inflation than the other surface fields for most of the first three-hours of the DA period (Fig. \ref{infstats}). Afterward, the domain maximum Q2 prior inflation is comparable to the values for other surface fields, likely due to the development of convection after 21Z (since this is a domain maximum, the largest values of inflation are likely associated with convectively-generated cold pools). An increase in the prior inflation for all the surface fields occurs after 22 UTC associated with a large increase in convection within the northern half of the domain. Interestingly, during this period, the Q2 domain-maximum prior inflation values are less than the other surface fields, with U10 and V10 having the largest values. This suggests errors in the low-level wind fields are larger than 2-m temperature and moisture errors within the developing cold pools.

\begin{figure}
\centering
\includegraphics[scale=0.6]{inflationstats}
\caption{Domain-maximum inflation values for 8 state fields from SFCRAD5H.}
\label{infstats}
\end{figure}

The 4 microphysical-related inflation variables increase with the onset of convection shortly after 21 UTC. During the first 45 minutes, the domain-maximum values are similar, but after 21:45 UTC, the domain maximum prior inflation stabilizes for QRAIN, while the values for QSNOW and QGRAUP continue to increase. Greater uncertainty exists in the prediction of the ice species compared to rainwater (likely due to both model error and forward operator error), and adaptive inflation accounts for this by inflating QSNOW and QGRAUP more than QRAIN. In this way, the behavior of the inflation algorithm is able to provide estimates of relative error between state fields, providing additional information to complement the observation-space innovation diagnostics. Examination of the spatial distribution and other aggregated statistics for inflation will likely provide further understanding on the nature of model error. This task is left for future work.

\paragraph{Implications for warn-on-forecast systems}
All of the above factors have implications in proposed future warning systems that use short-term forecasts [\(O(1hr)\)] from ensembles of convective-allowing simulations to provide intensity, track, and uncertainty guidance for forecasters \citep{stensrudetal09a,stensrudetal13}. While the model and observation resolutions in the present set of experiments are too coarse to fully resolve the processes that forecasters are most interested in (e.g. low-level mesocyclones, downdrafts), the one-hour forecasts of convection in SFCRAD5H were generally quite good. Even beyond one-hour of lead-time, the forecasts exhibited skill at providing valuable track information for individual storms (e.g. SFCRAD5H provided an extremely accurate forecast of the southern supercell in CI3).  Yet, even at higher resolutions, the forecasts of track and intensity will remain sensitive to the mesoscale environment in which convection  is embedded. Further, model errors will play a role in the successful assimilation of surface and radar DA in any simulation that makes use of parameterized processes important to convection.