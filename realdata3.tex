\section{Surface and radar data assimilation experiments}
Radar data were assimilated for two hours (21 UTC to 23 UTC) following the three hours of surface DA. Surface DA is also performed during this period. Reflectivity, radial velocity, and clear-air reflectivity observations were assimilated. The analyses and forecasts from this experiment (SFCRAD5H) will be analyzed in the following section.
\subsection{Observation-space diagnostics}
\subsubsection{Domain-averaged statistics}
Observation-space diagnostics were computed to assess the fidelity of the assimilation of radar observations, particularly observations of radar reflectivity that have been assimilated with varying degrees of success in previous work. Within this section, diagnostics are computing using observations that were successfully assimilated (and not those that were rejected). During the two hours of radar DA, radial velocity observations are fairly well-behaved (Fig. \ref{sawtooth}a). The posterior RMSI quickly stabilizes to just under 2 m s\textsuperscript{-1}, slightly larger than the prescribed observation error standard deviation. The prior total spread and prior RMSI behave consistently, especially by the end of the assimilation period, resulting in prior consistency ratios near 1. After the first 30 minutes of DA, bias (i.e. mean innovation) values approach zero, although after 22:15 UTC, bias values become slightly positive. This occurs during a large increase in the number of assimilated observations, and will be investigated section \ref{biassection}.

\begin{figure}
\centering
\includegraphics[scale=0.8]{radarDA_sawtooth}
\caption{Observation-space diagnostics for (a) radial velocity and (b) reflectivity observations in SFCRAD5H. Only those observations that were assimilated are used to compute the diagnostics. RMSI (solid black), mean innovation (solid green), and total spread (dotted black) are plotted in the top part of (a) and (b). The prior and posterior values are plotted together, resulting in the sawtooth pattern. The number of observations assimilated (gray circles) and rejected (red circles) at each assimilation time are plotted in the bottom half of (a) and (b).}
\label{sawtooth}
\end{figure}

Reflectivity RMSI gradually decreases during the first hour of assimilation, with the prior RMSI stabilizing to just below 6 dBZ by 22:30 UTC (Fig. \ref{sawtooth}b). The more gradual fit to reflectivity observations, and a larger RMSI, compared to radial velocity is likely due to the larger prescribed value of observation error for reflectivity (5 dBZ error variance). Both the prior and posterior total spread magnitudes are smaller than RMSI during the entire assimilation period, although this is improved later in the period. Reflectivity biases are initially large and negative as convection is spun-up in the ensemble analyses, but stabilize near -3 dBZ for the second half of the assimilation period. This persistent negative bias between 22 and 23 UTC will be investigated in section \ref{biassection}.

\subsubsection{Vertically-averaged statistics}
To assess variations of the statistics in the vertical, reflectivity and radial velocity observations were aggregated into 500-m vertical bins. Vertical RMSI, spread, and bias profiles for the final prior analysis time (23 UTC) illustrate some of these variations (Fig. \ref{vertprof}). Radial velocity RMSI is smallest just above the surface between 1 and 4 km (approximately 2.5 m s\textsuperscript{-1}; Fig. \ref{vertprof}a). Above 4 km, RMSI increases to approximately 3 m s\textsuperscript{-1}. Errors increase rapidly above 10 km due to the small number of observations and a reduced ability to observe the radial wind components. The total spread is larger than the RMSI at all vertical bins below 4 km, while there is a better correspondence between the total spread and RMSI between 4 km and 10 km. A small positive bias exists below 6 km, reaching a maximum near the surface with a local maximum between 3 and 4 km. A negative bias is present aloft between 11 km and 12 km.

\begin{figure}
\centering
\includegraphics[scale=0.6]{radarDA_vertprof}
\caption{As in Fig \ref{sawtooth}, but only for the prior analyses, with statistics aggregated into 500-m height bins. The environmental melting level from the 00 UTC 30 May 2012 KOUN sounding is denoted by the blue horizontal line.}
\label{vertprof}
\end{figure}

The largest reflectivity RMSI occurs at upper-levels (above 10 km where RMSI increases to \textgreater 10 dBZ) and near the surface (RMSI approximately 8 dbZ; Fig. \ref{vertprof}b). A minimum is reached in mid-levels between 4 and 6 km. A relative RMSI maxima is also present near 3 km. Total spread is in better agreement with RMSI below 6 km, although discrepancies do exist, most notably near the surface and between 2 and 3 km. Total spread is consistently smaller than RMSI above 6 km, with differences between the two growing with height. These differences in RMSI are strongly affected by the bias. Only the part of the profile between 1 and 3 km is near zero. A +5 dBZ bias is present in the lowest bin of reflectivity observations, and a +2.5 dBZ bias exists in the layer between 3 and 4 km. Above 4 km the bias becomes negatives and its magnitude increases with height. Reflectivity biases above 10 km increase to nearly 8 dBZ. These reflectivity biases are the primary reason RMSI also increases above 6 km. Understanding these systematic biases in the vertical for radial velocity and reflectivity will be the focus of the following section.

\subsection{Sources of analysis bias}
\label{biassection}
Prior state observation-space biases can exist for various reasons including instrument miscalibration, model error, including errors in model physics parameterizations, and errors in the forward operator connecting the model state to the observations. Since the DA update assumes both unbiased observations and forecasts, conducting DA with these biases present results in a suboptimal data assimilation update and leads to posterior analyses that possess the biases of the forecasts and observations. Thus, interpreting these biases should provide a deeper understanding of analysis and forecast quality.

\subsubsection{Radial velocity biases}
The vertical profiles of bias described in the previous section are plotted for each prior analysis time to gauge temporal trends in bias through the assimilation period (Fig. \ref{bias2d}; the column at 23 UTC corresponds to Fig. \ref{vertprof}). The first hour of DA (21 UTC to 22 UTC) will be disregarded since the statistics from the previous section indicated the DA system takes approximately this length of time to stabilize. Beyond this spin-up period, \(V_r\) biases are near zero through most of the vertical column except in two areas: above 11 km MSL and near the surface between 0.5 and 3 km MSL. In the former case, a negative \(V_r\) bias exists, due to a small number of \(V_r\) observations and a limited ability to retrieve the wind field at upper-levels. The latter \(V_r\) bias develops after 22:15 UTC and is positive, continuing through the end of the DA period, although reducing in magnitude after 22:45 UTC. While the overall distribution of \(V_r\) innovations is approximately Gaussian (Fig. \ref{histinnov}a), the positive \(V_r\) bias at low-levels results in a distribution mean near 1 m s\textsuperscript{-1}. 

The positive \(V_r\) bias is largest in the 1-2 km MSL layer with a magnitude between 2 -- 2.5 ms\textsuperscript{-1}. The \(V_r\) observations with the largest positive innovations within this layer are associated with the convection that developed in CI2 between 22:25 UTC and 22:55 UTC (Fig. \ref{innovlocvr}). Many \(V_r\) innovations are \textgreater\, 8 ms\textsuperscript{-1} in this region, leading to the positive bias in Fig. \ref{bias2d}a. These innovations are located within and between two storms that merge during the final 30-minutes of the DA period. The motion associated with the northern storm (storm N in Fig. \ref{innovlocvr}) is toward the SE, while the southern storm (storm S in Fig. \ref{innovlocvr}) motion is toward the NE; innovations greater than 8 m s\textsuperscript{-1} are present along the northeastern edge of storm S around 22:45 UTC (Fig. \ref{innovlocvr}c), suggesting that the model is unable to capture the low-level wind field associated with this storm merger.

\begin{figure}
\centering
\includegraphics[scale=1.0]{radarDA_bias2d}
\caption{Ensemble mean prior bias for (a) radial velocity and (b) reflectivity aggregated by height into 500-m bins for each assimilation time. The environmental melting level from the 00 UTC 30 May 2012 KOUN sounding is denoted by the blue horizontal line.}
\label{bias2d}
\end{figure}
\begin{figure}
\centering
\includegraphics[scale=0.9]{radarDA_histinnov}
\caption{Histogram of innovations for (a) radial velocity and (b) reflectivity for observations between 22:30 UTC and 23:00 UTC.}
\label{histinnov}
\end{figure}
\begin{figure}
\centering
\includegraphics[scale=1.0]{radarDA_innovloc_vr}
\caption{Radial velocity observations between 1 km and 2 km MSL at (a) 22:25 UTC, (b) 22:35 UTC, (c) 22:45 UTC, and (d) 22:55 UTC color-coded by the prior innovation associated with each observation. CREF \textgreater 40 dBZ is shaded.}
\label{innovlocvr}
\end{figure}
\begin{figure}
\centering
\includegraphics[scale=0.9]{radarDA_scatter}
\caption{Scatter plot of radial velocity observations at 22:45 UTC between 1 km and 2 km MSL and their associated prior mean value (same observations as shown in Fig. \ref{innovlocvr}c). Each point is color-coded by the innovation associated with each observation (prior mean value - observed value).}
\label{scatter}
\end{figure}

Most of the innovations \textgreater 8 m s\textsuperscript{-1} occur when the radial velocity is negative (i.e. away from the radar; Fig. \ref{scatter}), which is consistent with the storm motion away from the nearest radar (KVNX) during the storm merger. The prior mean forecast values associated with the positive innovations \textgreater 8 m s\textsuperscript{-1} fall into two regimes. First, prior forecast values that had the same sign as the observations (negative), but had a smaller magnitude. That is, the prior forecast radial velocity magnitude was smaller than observed, but was in the correct direction. A majority of the largest innovations are due to this type of error. A smaller batch of innovations occured when the prior forecast radial velocity was in the opposite direction compared to the observation, while the magnitude is similar. In the first case, with cloud base near 1.5 km MSL, the model likely underpredicted the magnitude of the wind field in the outflow of storm N or S, while in the second case the direction was poorly predicted, potentially due to a displacement error in the position of the surface cold pools associated with these merging storms. Given the complex nature of the convective evolution within CI2, and the transient nature of the bias (the positive radial velocity bias in this layer decreases to near zero by 23 UTC), it is hypothesized that the positive bias is mainly due to predictability limitations, with the true evolution of the cell merger falling outside the envelope of solutions within the ensemble, resulting in a positive radial velocity bias for a brief period of time (similar in nature to the large mean innovations present at the beginning of the DA period due to model spin-up).

\subsubsection{Reflectivity biases}
The vertical structure of \(Z\) bias is consistent between 22 UTC and 23 UTC (thus the vertical bias profile at 23 UTC presented earlier in this section is characteristic of the profile through this hour-long period). Positive mean \(Z\) innovations occur near the surface and between 3 and 4 km MSL, while above 6 km MSL, the mean \(Z\) innovations is negative, with increasing magnitude with height. Mean innovations in each of these regions will be discussed in turn below.

\paragraph{Surface \(Z\) bias}
A positive \(Z\) bias is present near the surface (below 1 km AGL) starting around 22:15 UTC. The magnitude of this bias varies through the rest of the DA period, but at several points in the DA period approaches +4 dBZ. This surface \(Z\) bias is largest for \(Z\) observations between 10 -- 20 dBZ, where the magnitude is \textgreater than 8 dBZ by 23 UTC (Fig. \ref{bias2d_zbins}a). The positive \(Z\) bias extends through a deeper layer for smaller \(Z\) observations (Fig. \ref{bias2d_zbins}a), and shallower layers larger \(Z\) observations (Fig. \ref{bias2d_zbins}b-d).

A plausible explanation for this bias is due to error from the double-moment microphysics scheme used in the forecast model. Double-moment schemes allow for hydrometeor sedimentation (the tendency for the mean hydrometeor size to increase toward the ground due to terminal velocity differences). This “size-sorting” process, while usually a transient phenomenon, can be sustained in environments with wind shear \citep{kumjianryzhkov12}. \citet{milbrandtyau05} described the tendency for double-moment microphysics schemes to overestimate mean rain drop-sizes near the surface due to the size sorting process (see their Fig. 3t). This leads to a positive bias in reflectivity (their Fig. 3n). To counter runaway growth of drop sizes near the surface, some double-moment schemes have placed an upper limit on mean drop diameter, or incorporated a drop-breakup parameterization that acts on drops larger than a prescribed threshold. While the microphysics scheme used herein employs the latter process, uncertainties exist in the formulation and choice of breakup threshold \citep{morrisonetal12}. Thus it seems reasonable to suggest a small positive bias of reflectivity near the surface is due to size-sorting. Radar echoes produced by anomalous propagation (i.e. ground clutter) could also produce a bias near the surface, but is likely not a culprit for the surface reflectivity bias since clutter would produce a negative bias (i.e. observed reflectivity larger than forecast reflectivity).

\paragraph{Mid-level \(Z\) bias}
The positive \(Z\) bias between 3 and 4 km occurs immediately below the environmental melting layer (Fig. \ref{bias2d}b), which is approximately located at 4.3 km AGL (determined from 00 UTC 30 May OUN radiosonde). Similar to the surface \(Z\) bias, this bias develops after 22:15 UTC and ranges between +1 and +3 dBZ during the second hour of DA. The largest bias occurs with \(Z\) observations between 10 and 30 dBZ (Fig. \ref{bias2d_zbins}b-d), since the reflectivity within the bright band typically falls within this range, although it is also present to some degree for all reflectivity observations between 10 and 40 dB (Fig. \ref{bias2d_zbins}a-d). Given the large negative bias for most \(Z\) observations \textgreater\, 40 dBZ, this bias near the melting layer is negative, but is 4 -- 6 dBZ smaller in magnitude (multiple sources of bias are present that reduce the bias for these observations).

\begin{figure}
\centering
\includegraphics[scale=0.6]{radarDA_bias2d_zbins}
\caption{As in \ref{bias2d}, but for reflectivity observations between (a) 10 -- 20 dBZ, (b) 20 -- 30 dBZ, (c) 30 -- 40 dBZ, and (d) \textgreater 40 dBZ.}
\label{bias2d_zbins}
\end{figure}

Given its proximity to the melting layer, this positive bias is likely related to the process of melting ice (both snow and graupel) that occurs in the layer below the melting level. While the microphysics scheme incorporates the physical processes of melting and the transition of mass between species that results from the melting process, it does not provide information pertinent to the computation of reflectivity within the melting layer, namely the amount of meltwater that is retained on the particle exterior during melting, compared to the amount that is absorbed into the interior of the particle. The effect of meltwater on the surface of the ice particle produces the characteristic “bright band” in radar imagery. Since a break exists between the levels where the largest \(Z\) bias exists at mid-levels and near the surface, the surface \(Z\) bias is not due to melting graupel that extends to the ground (in addition, vertical profiles of graupel mixing ratio indicate that little to no graupel reaches the ground).
	
The increase in reflectivity due to ice melting is modeled in the forward operator for reflectivity using the microphysical mixing ratios and temperature profile (see Appendix B for details). Yet, many assumptions must be made when computing reflectivity from melting ice particles, including the fraction of meltwater that is left on the surface and the amount of meltwater that is absorbed into the particle during the melting process. These fairly simple assumptions are associated with significant uncertainty and are likely responsible for the positive bias in reflectivity. This bias gives an indication of the degree of error present in the melting assumptions within the forward operator. One option is to reduce the fraction of meltwater left on the exterior to reduce the predicted reflectivity, or implement a more complex model for melting in the forward operator such as those found in \citet{jungetal10} or \citet{dawsonetal13}. These tests are left to future work.

\paragraph{Upper-level \(Z\) bias}
A large negative \(Z\) bias exists above 6 km (Fig. \ref{bias2d}b). This bias is dominated by 10 –- 20 dBZ reflectivity observations (Fig. \ref{bias2d_zbins}a); most of the observations above 6 km are within this range and are within the convective anvils. For example, the bias above 6 km for 20 –- 30 dBZ observations is generally much closer to zero. For higher observed values of reflectivity, fewer observations are present in upper-levels thus bias estimates are often noisy (Fig. \ref{bias2d_zbins}c-d). As was performed for the radial velocity innovations, the locations of observations from 23 UTC were plotted for observations within the 8 –- 10 km layer with values of reflectivity between 10 –- 20 dBZ (Fig. \ref{vrinnov}). Within this layer, most observations are located within the anvil region, which spreads toward the ESE away from the convective cores. Specifically, the observations with the largest negative innovations (greater than 8 dBZ) are located along the outer edge of the anvils, and result in a bimodal peak in the distribution of reflectivity innovations (\ref{histinnov}b). Here, the negative bias indicates that the anvils associated with convection in the prior forecasts have not spread as far downstream as the observations indicate.

\begin{figure}
\centering
\includegraphics[scale=0.8]{radarDA_innovloc1}
\caption{As in Fig. \ref{innovlocvr}, but for reflectivity observations between 8 km and 10 km MSL at 23:00 UTC.}
\label{vrinnov}
\end{figure}

One hypothesis for the reflectivity bias is a slow bias in the environmental wind profile. Due to this bias, storm motions were slower than observed in the forecasts and, at anvil-level, cloud ice and snow were not advected far enough downstream during the forecast. This is supported by the ensemble hodographs in CNTL and SFC3H valid at 23 UTC 29 May 2012 from the nearest grid point to the OUN sounding location (Fig. \ref{hodo}). The ensemble mean wind speed within the 8 –- 10 km layer varies from 18 m s\textsuperscript{-1} to 27 m s\textsuperscript{-1} at 8 km MSL and 10 km MSL, respectively, while the observed wind speed is 22 m s\textsuperscript{-1} at 8 km MSL and 31 m s\textsuperscript{-1} at 10 km MSL. Thus, the ensemble mean wind speed is approximately 4 m s\textsuperscript{-1} less than the observed wind speed within this layer (the differences in the forecast and observed wind direction are less significant). 

Assuming these differences between the forecast and observed profiles are similar across the domain, the integrated effects of this forecast error during the two-hour DA period could produce the large negative reflectivity biases along the downstream edge of the convective anvils. After 2 hours of radar DA, the 4 m s\textsuperscript{-1} anvil-level flow bias would lead to an upstream displacement error of approximately 30 km; this is similar to the width of the downstream anvil edge composed of the largest negative innovations.

A second hypothesis involves an underprediction of anvil-level ice and snow due to incorrect particle fall speeds within the microphysics parameterization. That is, as snow is advected downstream by the anvil-level winds, it tends to fall out more quickly than what is observed, underpredicting the snow mixing ratios leading to smaller than observed reflectivity. This is partly supported by the gradual increase in reflectivity bias that occurs in the anvils of the northern convection within CI2. Here, the reflectivity bias is actually positive closer to the convection core and becomes more negative further downstream within the anvil. A definitive explanation for these errors are left to future work, but its impact, and the impacts of the other reflectivity biases on optimal data assimilation, will be discussed in the following section.

\subsubsection{Discussion}
The various biases described in the previous sections can produce undesirable and deleterious effects on the resulting analyses and forecasts. For example, in the present experiments, the positive bias associated with the reflectivity forward operator within the melting layer results in prior innovations that are predominantly positive. For reflectivity, this implies a positive adjustment to the rain, snow, and/or graupel mixing ratios since reflectivity observations and hydrometeor mixing ratios are positively correlated. Since the bright band develops only once appreciable ice hydrometeors develop, these errors likely play a less significant effect early in the development of convection(Fig. \ref{bias2d}b). But once the melting process begins to occur within the forecast, assimilation of reflectivity observations, through positive correlations with the microphysical state, will result in an addition of mass to the hydrometeor species, solely due to the biased forward operator. In a cycled assimilation system such as the one used herein, these errors are propagated to the next assimilation time. Thus, a persistent continual increase in hydrometeor mixing ratios will continue to occur at every assimilation time within and surrounding the layer possessing melting ice hydrometeors. A similar behavior can occur with errors due to a biased forecast model.

\subsection{Ensemble forecasts from SFCRAD5H}
While the focus of the previous section was on the fidelity of the SFCRAD5H prior analyses produced during the 21 UTC to 23 UTC DA period, this section will examine the 50-member 3-hour ensemble forecasts initialized with the final 23 UTC posterior analyses. The ensemble forecasts provide an additional means to assess analysis quality, as reduced analysis error is expected to translate into gains in forecast skill.

For verification, a similar approach is taken to that applied with the surface DA experiments. This includes comparing forecast CREF areas with observed CREF by using the grid-point maximum value of CREF computed over a one-hour period or the entire three-hour forecast period. Probabilities of CREF exceeding 25 dBZ between 23 UTC and 02 UTC are generated with these 1-hr and 3-hr maximum fields (PROB1H-CREF25 and PROB3H-CREF25, respectively). This same procedure was applied to the observed CREF dataset, although for these comparisons an observed CREF threshold of 40 dBZ is applied to isolate the track of each convective core (since convection has matured by 23 UTC, using the observed 25 dBZ CREF contour would include large areas of anvil).

To identify the tracks of the most intense convective cores in SFCRAD5H, the maximum upward vertical velocity (UVV) between the surface and 400 hPA is recorded at each time-step and a time-maximum at each grid point is applied to produce an hourly-maximum UVV field (analogous to the hourly-maximum CREF field). Hourly-maximum UVV is used in addition to hourly-maximum simulated CREF, including both one-hour and three-hour UVV probabilities (PROB1H-UVV10 and PROB3H-UVV10, respectively) and the one-hour and three-hour ensemble maximum UVV (MAX1H-UVV and MAX3H-UVV, respectively). Verifying the probabilistic forecasts is challenging since  only one event is analyzed, thus, the comparisons between forecasts and observations will be subjective. Further, analysis will  focus on the three areas of most intense convection in the domain (the southeastward moving supercells in north-central OK, the left-split and right-split originating within CI3; Fig. \ref{radarsummary})

\begin{figure}
\centering
\includegraphics[scale=0.85]{radarDA_3hprob}
\OUsinglespace
\caption{23 UTC to 02 UTC probability of (a) CREF \textgreater 25 dBZ (PROB3H-CREF25) and (b) UVV \textgreater 10 m s\textsuperscript{-1} (PROB3H-UVV10). The areas where one-hour maximum observed CREF \textgreater 40 dBZ is denoted by the solid black contour.}
\label{3hrprob}
\end{figure}

Overall, the SFCRAD5H forecast fields of PROB3H-CREF25 and PROB3H-UVV10 capture the most intense observed convection within the domain (Fig. \ref{3hrprob}). In the northern half of the domain, a NW-SE axis probability maximum is collocated with the track of several supercells that propagated toward the SE between 00 UTC and 02 UTC. These storms developed within the southern portion of CI2 and produced very large hail and a brief tornado, and were among the most intense storms during this event. The orientation and magnitude of the probability axis provides confidence in the corridor where these supercells would eventually track during the forecast period. Within this corridor, convection is forecast to move slower than observed, likely due to the previously mentioned environmental wind speed bias (Fig. \ref{hodo}). For example, by 02 UTC the observed right-moving supercell was in the process of merging with the left-moving storm originating in CI3, but the PROB3H-CREF25 axis suggests that no members forecast this merger to occur before the end of the forecast period. This bias affects, to some degree, all the convective-scale forecasts in this chapter and illustrates how mesoscale errors can limit convective-scale forecast skill. The PROB3H-UVV10 swath is almost perfectly co-located with the observed CREF25 contour (Fig. \ref{3hrprob}). Timing errors for this storm are less than other areas of convection. Given its predicted intensity, a slower storm motion due to internal storm processes likely led to smaller three-hour forecast errors \citep{bunkersetal00}.

SFCRAD5H also does a poor job forecasting the development of a line of convection in NE OK after 01 UTC. This line grows upscale and is responsible for severe wind gusts across eastern and southeastern OK after 02 UTC, but this portion of the case will not be the focus of any analysis since none of the experiments conducted herein were able to capture its development, although several members do predict its development after 02 UTC. Future work should investigate the ability of radar DA beyond 23 UTC to capture and predict this part of the convective system.

\begin{figure}
\centering
\includegraphics[scale=0.9, angle=90]{radarDA_1hprob_cref}
\caption{As in Fig. \ref{3hrprob}a, but for one-hour probabilities of CREF \textgreater 25 dBZ (PROB1H-CREF25) and one-hour maximum observed CREF \textgreater 40 dBZ from (a) 23 UTC to 00 UTC, (b) 00 UTC to 01 UTC, and (c) 01 UTC to 02 UTC.}
\label{1hprob_cref}
\end{figure}
\begin{figure}
\centering
\includegraphics[scale=0.9, angle=90]{radarDA_1hmax_uvv}
\caption{One-hour ensemble maximum UVV (MAX1H-UVV; m s\textsuperscript{-1}) and one-hour maximum observed CREF \textgreater 40 dBZ from (a) 23 UTC to 00 UTC, (b) 00 UTC to 01 UTC, and (c) 01 UTC to 02 UTC.}
\label{1hmax_uvv}
\end{figure}

The PROB1H-CREF25 field between 23 UTC and 00 UTC demonstrates the effectiveness of radar DA at the beginning of the forecast period (Fig. \ref{1hprob_cref}a). Higher values of PROB1H-CREF25 (\textgreater 95\%) are colocated with areas of active convection, indicating radar DA was successful at sustaining convection during the first forecast hour. Two areas exist where convection appears to be poorly initialized. The southern storms within CI2 are associated with lower PROB1H-CREF25 values during the first hour than the convection to its north. Also, in CI3, the left-split associated with the northern storm in CI3 is associated with smaller PROB1H-CREF25 values than the southern storm. Several ensemble members capture the motion and track of the left-split during the 23 UTC -- 00 UTC period, although by the end of the one-hour forecast, the observed storm location is ahead of the envelope of CREF probabilities. In CI4, PROB1H-CREF25 values are reduced compared to the surface DA experiments (c.f. Fig \ref{sfc3hcrefprob}b and Fig. \ref{3hrprob}a), due to the assimilation of clear-air observations (non-zero probabilities remain near DFW in a region not affected by clear-air radar observations).

Between 00 UTC and 01 UTC, PROB1H-CREF25 values increase in the area from CI1 to CI2. Here, three probability maxima are present associated with the following: 1) convection originating in CI1 that moves toward the SE, 2) convection in the northern part of CI2 that moves eastward, and 3) new convective development just west of CI2, between CI1 and CI2 (Fig. \ref{1hprob_cref}b and \ref{1hmax_uvv}b). The southeastern part of the third area is composed of convection that developed in the southern part of CI2, along with new development to its west. This convection is responsible for extending the PROB1H-CREF25 axis further to the southeast compared to the forecasts that only assimilate surface data. SFCRAD5H also captures the convective development in south-central KS. Further south, PROB1H-CREF25 values \textgreater 90\% are co-located with the southern right-moving supercells in CI3, while lower PROB1H-CREF25 values are present in association with the northern storm. SFCRAD5H successfully predicted that the southern storm would intensify after 00 UTC, with MAX1H-UVV values greater than 50 m s\textsuperscript{-1} associated with this convection (Fig. \ref{1hmax_uvv}b). Further, the demise of the northern right-moving storm in CI3, once the southern storm intensified, is well predicted. A third of the ensemble successfully predicts the occurrence of a left-moving supercell, as indicated by PROB1H-CREF25 values \textgreater 35\% extending to the NE toward OKC between 00 UTC and 01 UTC (Fig \ref{1hprob_cref}b), along with MAX1H-UVV storm tracks (Fig. \ref{1hmax_uvv}b). This left-moving storm is observed to move faster than predicted by the swaths of both PROB1H-CREF25 and MAX1H-UVV.

After 01 UTC, PROB1H-CREF25 values increase within north-central OK (Fig. \ref{1hprob_cref}c). Again, this convection moves slower than the observed convection between 01 and 02 UTC, likely due to errors in the cloud-layer environmental wind field discussed previously. The highest PROB1H-CREF25 values (\textgreater 80\%) in CI3 are associated with the right-moving supercell in northern TX. The position of this storm between 01 UTC and 02 UTC is predicted well. In CI3, the PROB1H-CREF25 field extends further downstream than the observed CREF \textgreater 40 dBZ contour due to the former field capturing the convective anvil. The left-split in CI3 is forecast by approximately a third of the ensemble members to continue moving northeastward through 02 UTC (Fig. \ref{1hprob_cref}c).

\subsection{Ensemble forecasts from 22 UTC, 22:30 UTC, 23 UTC}
\label{inittime_section}
Ensemble forecasts were initialized at 22 UTC, 22:30 UTC, and 23 UTC after 1, 1.5, and 2 hours of radar DA (in addition to surface DA), respectively. As discussed in section \ref{sfcDA_diss}, differences between these experiments are due to a combination of the longer DA period and the shorter-forecast lead-time. Yet, the effects of radar DA can be seen through the systematic reorientation of forecast probabilities as radar data are assimilated for longer periods.

\begin{figure}
\centering
\includegraphics[scale=0.9,angle=90]{radarDA_init22}
\caption{As in Fig. \ref{3hrprob}b, but for ensemble forecasts initialized at (a) 22 UTC, (b) 22:30 UTC, and (c) 23 UTC.}
\label{init22}
\end{figure}

As the DA period is extended, PROB3H-UVV10 values increase along the axis of the most intense observed convection across central OK (Fig. \ref{init22}). While this axis extends further toward the SE in the 23 UTC forecast, this axis of highest PROB3H-UVV10 values is slightly biased toward the NW, with convection not reaching the OKC metropolitan area until after 02 UTC (in reality this occurred closer to 01 UTC). In areas surrounding the observed CREF \textgreater 25 dBZ contour, PROB3H-UVV10 values are reduced as additional clear-air observations are assimilated. Since CI occurred between 21:15 UTC and 21:30 UTC, the forecast initialized at 22 UTC has only been impacted by 30-45 minutes of radar DA (not including clear-air observations), during the early evolution of the storms.

Differences between the three forecasts are largest in CI3. In the forecast initialized at 22 UTC, the PROB3H-UVV10 swaths indicate the development of a left- and right-moving pair of storms originating near the position of the observed northern supercell (Fig. \ref{init22}a). Relatively low PROB3H-UVV10 values (\textless 35\%) exist along the path of the observed southern supercell. In the 22:30 UTC forecast, PROB3H-UVV10 values increase along the observed track of the southern supercell (Fig. \ref{init22}b), and by 23 UTC, this axis narrows and probabilities increase to \textgreater 80\% (Fig. \ref{init22}c). Further, given the agreement among the ensemble members in the forecast track of the southern supercell in CI3 by the 23 UTC forecast, nearby areas of small PROB3H-UVV10 values that are present in the 22 UTC and 22:30 UTC forecasts are reduced in the 23 UTC forecast, especially in areas west of the track, with PROB3H-UVV10 values decreasing to \textless 20\%, in the region near CI4.

The forecast of the observed left-split that emerges from CI3 is sensitive to the initialization time. In the 22 UTC forecast, the northern supercell was long-lived and the southern supercell was not predicted as confidently (Fig. \ref{init22}a), resulting in more members that sustain a left-split originating from the northern supercell. In the 23 UTC forecast, the southern supercell was predicted to be long-lived, while less than half of the ensemble members successfully forecast the development of the northern supercell (Fig. \ref{init22}c). This change in forecast behavior leads to fewer left-splits in the ensemble along the axis where the left-split was observed in the forecast initialized later, even after two hours of radar DA. The differences between the ensemble forecasts suggests that DA has difficulty in developing both the northern and southern supercell in CI3 as was observed, leading to subsequent differences in the predictions of the long-lived left-split. Due to the juxtaposition of these two storms, limitations from the relatively coarse model and observation resolution in these experiments may hamper the ability to accurately predict both storms.

\subsection{Impact of surface and radar data on 0-3hr forecasts}
Two additional experiments were conducted to isolate the effects of the radar and surface DA. In the first, SFC3H is extended by assimilating surface data every 5-minutes for an extra two hours between 21 UTC and 23 UTC (SFC5H). In the second, surface data are withheld; radar DA begins at 21 UTC and continues to 23 UTC (RAD2H), thus the initial 18 UTC ensemble is advanced freely to 21 UTC before radar DA, without any surface DA. SFCRAD5H, SFC5H, RAD2H, and CNTL will be compared using the PROB3H-UVV10 field. Differences between SFC5H (Fig. \ref{3hprobuvv4expN}b) and SFCRAD5H (Fig. \ref{3hprobuvv4expN}d) are primarily due to radar DA, while differences between RAD2H (Fig. \ref{3hprobuvv4expN}c) and SFCRAD5H (Fig. \ref{3hprobuvv4expN}d) are primarily due to surface DA. To map the response process in space-time, we ran queries on our coded transcripts for instances when a portion of text received codes for both a component of the response process (e.g., awareness, confirmation, protective actions) and any spatio-temporal benchmark.

\subsubsection{Northern half of domain}
When only surface data are assimilated, the PROB3H-UVV10 axis in central OK is centered to the NW of the observed axis of convection (Fig. \ref{3hprobuvv4expN}b) with the most intense convection straddling the observed CREF boundary (Fig. \ref{3hmaxuvv4expN}b). The maximum of PROB3H-UVV10 near CI1 matches well with the observed CREF area. When only radar data are assimilated (RAD2H), the axis of convection in central OK is centered further toward the SE (Fig. \ref{3hprobuvv4expN}c and Fig. \ref{3hmaxuvv4expN}c) compared to SFC5H. In addition, a bulls-eye of higher PROB3H-UVV10 values is present to the NE of the main PROB3H-UVV10 axis that is not present in SFC5H. Both of these improvements over SFC5H are due to the assimilation of radar observations associated with CI2. On the other hand, errors exist in RAD2H that do not exist in SFC5H. Convection in CI1 is forecast in RAD2H to persist well into the forecast period, long after this convection was observed to dissipate (Fig. \ref{3hprobuvv4expN}c and Fig. \ref{3hmaxuvv4expN}c). Also, the area of low-end PROB3H-UVV10 values (\textless 20\%) that fall outside the envelope of observed convection is larger in RAD2H, while the low-end PROB3H-UVV10 maximum in southern KS (near Wichita) associated with observed convection late in the forecast period is better placed in SFC5H. Thus, surface DA imparts a significant contribution to reducing these low-end probabilities, even more so than through the assimilation of clear-air observations.

\begin{figure}
\centering
\includegraphics[scale=0.75, angle=90]{radarDA_3hprob_uvv_4expN}
\caption{As in Fig. \ref{3hrprob}b, but for PROB3H-UVV10 values in the northern half of the domain from (a) CNTL, (b) SFC5H, (c) RAD2H, and (d) SFCRAD5H.}
\label{3hprobuvv4expN}
\end{figure}
\begin{figure}
\centering
\includegraphics[scale=0.75, angle=90]{radarDA_3hmax_uvv_4expN}
\caption{As in Fig. \ref{1hmax_uvv}, but for one-hour ensemble maximum UVV \textgreater 10 m s\textsuperscript{-1} (MAX1H-UVV) in the northern half of the domain from (a) CNTL, (b) SFC5H, (c) RAD2H, and (d) SFCRAD5H. Black contour is three-hour maximum observed CREF \textgreater 40 dBZ.}
\label{3hmaxuvv4expN}
\end{figure}

While radar DA alone does lead to forecast improvements, the RAD2H ensemble retains several detrimental environmental errors present within CNTL, thus RAD2H is handicapped without the addition of surface observations to correct errors in the surface environment, leading to increased convective forecast error compared to SFCRAD5H. The best forecast comes when both datasets are assimilated together (SFCRAD5H). Further, among the three experiments, the spatial pattern of PROB3H-UVV10 and MAX1H-UVV values in RAD2H is most similar to CNTL (c.f. Fig. \ref{3hprobuvv4expN}a,c), while the pattern in SFC5H is most similar to SFCRAD5H (c.f. Fig. \ref{3hprobuvv4expN}b,d). In other words, assimilating surface data alone provides gains in forecast accuracy (as primarily assessed by PROB3H-UVV10) that are at least as large, if not larger, than those gained by assimilating radar data.

\subsubsection{Southern half of domain}
The forecast of convection in the southern part of the domain is quite sensitive to the inclusion of the radar and surface datasets. Assimilating surface data alone, as in SFC5H, produces a northern supercell that is long-lived, leading to low, diffuse probabilities along the observed track of the southern supercell where only a few members produce convection (Fig. \ref{3hprobuvv4expS}b). With radar data alone, as in RAD2H, the southern storm is predicted to be dominant, leading to fewer members that produce long-lived convection where the northern storm was observed (Fig. \ref{3hprobuvv4expS}c). The combination of surface and radar data in SFCRAD5H produces a narrow swath of large PROB3H-UVV10 values (\textgreater 80\%) in excellent agreement with the observed track of the southern supercell (Fig. \ref{3hprobuvv4expS}d). The increase in PROB3H-UVV10 values is due to decreased spread in the forecast tracks of convection among the ensemble members (c.f. Fig. \ref{3hmaxuvv4expS}c,d). PROB3H-UVV10 values for the northern supercell are similar in SFCRAD5H and RAD2H. The left-split storm is more favored in SFC5H than RAD2H or SFCRAD5H for similar reasons as discussed in section \ref{inittime_section}. In SFC5H, more members agree in the development of a dominant northern supercell, resulting in more tightly clustered storm tracks compared to the RAD2H and SFCRAD2H, increasing PROB3H-UVV10 values (c.f. Fig. \ref{3hmaxuvv4expS}b,d) Finally, in the experiments that assimilated clear-air observations (RAD2H and SFCRAD5H), PROB3H-UVV10 values are reduced where convection was not observed, while in SFC5H, probabilities \textgreater 35\% remain in CI4, similar to CNTL.

\begin{figure}
\centering
\includegraphics[scale=0.9]{radarDA_3hprob_uvv_4expS}
\caption{As in Fig. \ref{3hprobuvv4expN}, but for PROB3H-UVV10 in the southern half of the domain.}
\label{3hprobuvv4expS}
\end{figure}
\begin{figure}
\centering
\includegraphics[scale=0.9]{radarDA_3hmax_uvv_4expS}
\caption{As in Fig. \ref{3hmaxuvv4expN}, but for MAX1H-UVV (m s\textsuperscript{-1}) in the southern half of the domain.}
\label{3hmaxuvv4expS}
\end{figure}

These results suggest that the biggest benefit to forecast skill is radar DA in the southern part of the domain, although the combination of surface and radar DA again produces the most skillful forecast in both parts of the domain. Surface DA plays an important role in improving short-term forecast skill in both portions of the domain, with a larger impact in the northern portion of the domain where errors in the prediction of surface fields and placement of boundaries are larger and result in larger errors in convective development between 00 UTC and 01 UTC.

\subsection{Impact of reflectivity assimilation on 0-3hr forecasts}
\label{section_zassim}
Given the reflectivity biases described in a previous section, and its non-linear relation to the state, it is possible that the assimilation of reflectivity observations is detrimental to the state estimate. To test the sensitivity of the forecasts to the assimilation of reflectivity, reflectivity observations in precipitation were withheld from the assimilation and a 3-hour 50-member ensemble forecast was initialized from the posterior ensemble at 23 UTC (denoted SFCRAD5H-NoZ). Radial velocity and clear-air radar observations were assimilated and aside from the lack of reflectivity observations the experiment is designed identically to SFCRAD5H.

\begin{figure}
\centering
\includegraphics[scale=0.9]{radarDA_3hprob_uvv_noZ}
\caption{As in Fig. \ref{3hrprob}b, but for PROB3H-UVV10 from (a) SFCRAD5H-NoZ and (b) SFCRAD5H.}
\label{probuvvnoZ}
\end{figure}

Reflectivity assimilation is especially beneficial for the forecast of the southern supercell originating within CI3 (c.f. Fig. \ref{probuvvnoZ}a,b), while in other areas of the domain, differences between SFCRAD5H and SFCRAD5H-NoZ are less substantial. Without reflectivity assimilation, the maximum PROB3H-UVV10 values along the predicted track of the southern supercell in CI3 are reduced by 30-40\%, and the probability axis is shifted toward the east. Also, as indicated by low-end probabilities, several members produce convection to the west of the storm track, expanding the area of non-zero probabilities outside of the observed track. While reflectivity assimilation is beneficial for this storm, the forecast is slightly improved in other areas of the domain without reflectivity assimilation. For example, the PROB3H-UVV10 forecast in SFCRAD5H-NoZ produces a more narrow PROB3H-UVV10 axis in central OK that extends further to the southeast compared to SFCRAD5H, in slightly better correspondence with the observed track of convection in this corridor. Overall, the addition of reflectivity observations is not substantially detrimental to the PROB3H-UVV10 forecast, but does play a role in improving the forecast for the isolated convection in CI3.

During the second half of the DA period, the southern supercell was observed to weaken slightly and then rapidly strengthen near 23 UTC. The assimilation of reflectivity data may help spin-up the storm more quickly during this secondary intensification, improving SFCRAD5H over SFCRAD5H-NoZ. In addition, since much of the storm is between 5 -- 15 km of the KFDR radar site, the portion of the storm above 5 km is only sampled by KTLX and KDYX, approximately 190 km and 200 km away from the storm, respectively. (at a range of 15 km the 19.5 degree beam, the highest radar tilt, is at 5 km AGL near the position of the southern supercell). At these ranges, the lowest tilts are spaced between 1.5 km and 2 km apart in the vertical, thus the impact of vertical correlations between radar data and the state is likely more important for this storm. Also, the large resolution volume for KTLX and KDYX radar observations likely leads to forward operator error at these ranges (the effects of the increased resolution volume are not accounted for in the forward operator). In other words, the reflectivity data from KFDR is important for the development of this storm by exploting the vertical correlations within the ensemble. This may help maintain the storm in the analyses as it passes near the radar site, during a period of apparent weakening. This hypothesis will be further discussed in the following section.

\subsection{Impact of radar localization on 0-3hr forecasts}
In Part I, synthetic radar observations of a simulated convective system were assimilated into a collection of OSSEs. The OSSEs from Part I varied only in the values prescribed for the horizontal and vertical covariance localization used for the radar observations. The experiments with a horizontal localization cutoff of 18 km and vertical localization cutoff of 3 km both improved analysis quality among a set of experiments that systematically adjusted the cutoff length over a range of values. Yet, for all of the experiments conducted so far in this section, the localization cutoff was set to 12 km in the horizontal and 6 km in the vertical, more in-line with previous work since it was uncertain how the OSSE results would apply to a real-data case study. To assess the sensitivity of the forecasts to these localization choices, SFCRAD5H was repeated, but with the horizontal localization increased to 18 km (SFCRAD5H-H18V6). In another experiment, the vertical localization was decreased to 3 km (SFCRAD5H-H12V3).

\begin{figure}
\centering
\includegraphics[scale=0.8,angle=90]{radarDA_3hprob_uvv_loc}
\caption{As in Fig. \ref{3hrprob}b, but for PROB3H-UVV10 from (a) SFCRAD5H-H12V3, (b) SFCRAD5H, and (c) SFCRAD5H-H18V6.}
\label{probuvvloc}
\end{figure}

Increasing the horizontal localization length produces several beneficial changes to the PROB3H-UVV10 forecast (c.f. Fig. \ref{probuvvloc}b and Fig. \ref{probuvvloc}c). Improvements include a larger area of probabilities \textgreater 80\% along the axis of forecast convection in central OK, and the elimination of the left-splits from CI3 in several members that were produced outside the observed track of this storm; the remaining members that produce a long-lived left-split do so along the observed path. Probabilities along the path of the southern supercell within CI3 are nearly unchanged between the two experiments. Decreasing the vertical localization length has a substantial negative impact on the forecast, primarily for the southern supercell in CI3 (c.f. Fig. \ref{probuvvloc}a and Fig. \ref{probuvvloc}b). Maximum PROB3H-UVV10 values associated with this storm are decreased to \textless 35\%, and the probability axis is more diffuse (Fig. \ref{probuvvloc}a). Forecast probabilities are also reduced in central OK, although the changes are smaller than those associated with the supercell in CI3.

These sensitivities suggest that permitting observations to update the state over a sufficiently deep layer can be important for producing accurate storm-scale analyses and short-term forecasts of convection in certain situations. In this case, the decrease of the vertical localization cutoff to 3 km in SFCRAD5H-H12V3 was detrimental to the analyses of convection for the supercell in CI3, as radar observations were unable to adequately update the state using the vertical covariance information. Given that the forecasts that did not assimilate reflectivity observations resulted in similar  forecasts to those in this section that (section \ref{section_zassim}), the added benefit of deeper vertical localization is primarily provided by reflectivity observations. Further, more modest differences in other areas of the domain when the vertical cutoff is decreased or when reflectivity is not assimilated, suggest this is not a universal finding. As summarized in section \ref{section_zassim}, the reasons for this behavior may be due to both the precise pattern of sampling by the three nearest WSR-88Ds given the storm's position and the apparent weakening of convection at the end of the DA period. 

\subsection{Summary and Discussion}
Results from several radar DA experiments were discussed in the present chapter. In the primary experiment, referred to as SFCRAD5H, radar and surface data were assimilated every 5-minutes during a two-hour period, beginning with the final set of SFC3H analyses at 21 UTC and ending at 23 UTC. Innovation diagnostics from SFCRAD5H revealed several biases in the prior ensemble analyses. These are summarized below, including a candidate explanation for the source of each error:

\begin{itemize}
\item Positive near-surface radial velocity bias due to model error during storm mergers.
\item Positive near-surface reflectivity bias due to excessive size-sorting in the Morrison double-moment microphysics scheme.
\item Positive mid-level reflectivity bias within the melting layer due to errors due to a simplified melting process in the forward operator.
\item Negative upper-level reflectivity bias due to a slow bias in the environmental anvil-level wind or microphysical errors in the handling of ice in the anvil layer.
\end{itemize}

Little work has been undertaken to explicitly diagnose the systematic errors that lead to suboptimal updates during radar DA. As discussed in section \ref{background_radarenkf}, \citet{dowelletal11} identified various sources of error, many due to the inherent inability of the microphysics parameterization to represent certain physical processes, leading to detrimental effects on the analyses during DA. In their experiments with a single-moment microphysics scheme, large negative biases occurred below 3 km AGL, in contrast to the positive bias noted herein near the surface (c.f. Fig. \ref{bias2d}b and their Fig. 4). The differences in near surface reflectivity bias between the present work and \citet{dowelletal11} are consistent with the deficiencies of single and double-moment microphysics schemes in producing reflectivity near the surface noted in other studies (e.g. \citealt{dawsonetal10,kumjianryzhkov12}). A small positive bias is also evident in mid-levels in Fig. 4 of \citet{dowelletal11}, yet this feature was not mentioned.

Using double-moment microphysics schemes and more complex forward operators during DA has only been conducted in a handful of studies. \citet{jungetal12}, \citet{putnametal13}, and \citet{yussoufetal13} all noted improvements in the analyses due to the ability of double-moment schemes to better reproduce various features of supercells and mesoscale convective systems over single-moment schemes. Radar emulators, which have mostly been used to validate simulations of convective storms (e.g. \citet{jungetal10,dawsonetal13}, have the potential to be used as a forward operator for DA (e.g. \citet{jungetal12}). Using the more advanced handling of melting ice in these emulators as part of the reflectivity forward operator would likely mitigate some of the bias present in the more simplified melting model used to compute reflectivity here.

Ensemble forecasts were initialized at 23 UTC from the final posterior SFCRAD5H analyses. The ensemble accurately predicted the short-term evolution of several regions of convection, including different types of convective modes. Various sensitivity experiments were conducted to assess the impact of surface and radar data (RAD2H and SFC5H), effectiveness of the reflectivity assimilation (SFCRAD5H-NoZ) and selection of horizontal and vertical localization for radar data (SFCRAD5H-H12V3 and SFCRAD5H-H18V6). A slow bias in the cloud-layer environmental wind profile in the analyses lead to timing errors in all of the forecasts. The axis of probabilities across central OK associated with the southeastward moving supercells, which produced the most intense severe weather, was largely insensitive to changes in the types of data assimilated (e.g. reflectivity) and variations in localization cutoff length. While part of the forecast probabilities in this region were due to convection that developed during the DA period within CI2, a larger amount of convection developed after 00 UTC. Since this occurs after the DA period, radar DA played a lesser role in 0-3hr forecast accuracy across the northern half of the domain. As discussed earlier, surface DA had a larger impact on the forecasts in this region, likely by adjusting surface boundaries (both mesoscale boundaries such as the dryline and convective-scale outflow boundaries associated with upscale growth), that led to more accurate short-term forecasts of convection.

In the southern part of the domain, the predicted convective evolution over the 3 hour forecast period in CI3 was more sensitive to the radar DA strategy. In experiments where radar data were withheld (SFC5H), low forecast probabilities were produced along the path of the observed southern supercell, while the northern storm was forecast to persist as a long-lived right-moving supercell in a majority of ensemble members. When surface data were withheld (RAD2H), the forecast probability swath was aligned with the observed path, but some ensemble members develop convection to the west of the observed storm, reducing the magnitude of forecast probabilities. Assimilating both surface and radar DA provided the most accurate prediction of the track of the long-lived southern supercell during the three-hour forecast. The storm timing errors during the three-hour forecast seen in other areas were smaller for the long-lived southern supercell, likely due to the supercell dynamics causing a slower storm motion \citep{bunkersetal00}.

Forecasts of the left-split originating from the northern supercell in CI3 were less sensitive to radar or surface DA, although radar DA slightly reduced the probability values due to less robust convection associated with the northern supercell. Finally, the assimilation of reflectivity data and deeper vertical localization cutoff (6 km vs. 3 km) both played a role in producing a better initial representation of the right-moving supercell in CI3, leading to a substantially more accurate 0-3 hr forecast. The left-moving storm emanating from CI3 was less sensitive to these variations, although spurious left-splits were suppressed in the experiment with larger horizontal localization.
