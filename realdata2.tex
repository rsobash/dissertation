\section{Surface data assimilation experiments}
\subsection{Experiment design}
Surface data observations were assimilated every 5-minutes between 18 UTC and 21 UTC. To assess the impact of various lengths of data assimilations, 50-member ensemble forecasts were launched at 19 UTC (SFC1H), 20 UTC (SFC2H), and 21 UTC (SFC3H). This encompasses CI and the early evolution of convection. These experiments are summarized in Figure \ref{sfcsummary}. In addition to the three ensemble forecasts, a control ensemble forecast (CNTL) was initialized with the 18 UTC initial conditions from the downscaled NCAR mesoscale ensemble. No DA was used within this control forecast. In the following sections, the emphasis will be on the differences in the forecast of CI and subsequent convective evolution during the 21 UTC to 00 UTC forecast period between the various experiments.

\subsection{Methods of forecast comparison}
\label{fcstcompare}
In both model output and observed radar data, convection was identified as areas where composite reflectivity (CREF) exceeded 25 dBZ (abbreviated as observed and forecast CREF25). This is a relatively liberal threshold, but was chosen to identify early signs of convective development and to identify convection in the ensemble mean, where CREF values may not exceed higher thresholds due to averaging among members, especially at longer forecast lead times. Several fields derived from the forecast CREF25 field were produced to evaluate differences between experiments and observations, including one-hour and three-hour ensemble probabilities of CREF25 (denoted PROB1H-CREF25 and PROB3H-CREF25, respectively). The PROB1H-CREF25 forecast was computed from the hourly-maximum CREF forecast, while the PROB3H-CREF25 forecast was computed from the three-hourly maximum CREF forecast derived with the individual hourly-maximum CREF fields over three consecutive hours. At each grid point, the fraction of members where CREF exceeded 25 dBZ, using the one or three-hour maximum CREF field, determined PROB1H-CREF25 and PROB3H-CREF25.

\begin{figure}
\centering
\includegraphics[scale=0.6]{surfaceDA_summary}
\caption{Summary of data assimilation experiments. Red lines indicate DA period, while black lines indicate 50-member ensemble forecasts initialized at different points during the DA period.}
\label{sfcsummary}
\end{figure}

In addition to the probabilities, two methods were developed to enable a more refined inspection of ensemble forecast CI timing errors. In the first method, areas where ensemble mean CREF exceeded 25 dBZ (denoted by MEAN-CREF25) were plotted at 15-minute intervals with earlier areas plotted on top of later areas. This provided a simple way to visually interpret ensemble mean timing errors. The second method used the 15-minute CREF output to compute the centroid of each CREF \textgreater 40 dBZ object within each ensemble member. These points are color coded in time to provide a sense of the variation among ensemble members in timing and spread of convection. A size threshold of 10 grid points was used to remove small CREF objects. The same procedure was applied to the observed CREF field to produce a verifying centroid track.

\subsection{CNTL ensemble forecast}
In this section, the CNTL forecast will be summarized to provide some perspective on the performance of SFC1H, SFC2H, and SFC3H. Non-zero values of PROB3H-CREF25 are clustered near the areas of convective development in the three CI regions. Within CI1, PROB3H-CREF25 values exceed 80\% (Fig. \ref{sfc3hcrefprob}a), with the spatial coverage of probabilities suggesting spread in the placement of convection. CNTL largely failed to initiate convection in CI2 during the three-hour period. Low values of PROB3H-CREF25 exist along the western edge of the observed CREF25 contour in CI2, associated with convection that develops late in the 21 UTC -- 00 UTC period. The convective development in CI3 was well captured in CNTL, depicting two distinct locations of CI, with slightly larger probabilities for the southern storm. The extension of probabilities to the north and east with the northern supercell suggests the development of northeastward moving convection (i.e. left-moving supercells).

CNTL also produced spurious convection in several regions, the most notable being near CI4 in north central TX. Here, a focused area of higher PROB3H-CREF25 values (\textgreater 65\%) exists where a majority of ensemble members predicted the development of a pair of right-moving and left-moving convection, a scenario that was not observed, although visible satellite imagery indicated attempts at convection near 21 UTC in proximity to the PROB3H-CREF25 maximum (Fig. \ref{satellite}a). Other areas of spurious convection in CNTL exist along the stationary boundary in southeast KS and in association with terrain across western AR. While several isolated, short-lived areas of convection were observed in these locations, PROB3H-CREF25 from CNTL appear to overestimate convective coverage.

The PROB1H-CREF25 and MEAN-CREF25 fields both reveal timing differences between forecast and observed CI in CNTL. Within CI1 and CI2, no ensemble members forecast CREF25 within the 21 UTC -- 22 UTC period, during the period when CI occurred (Fig. \ref{sfc1hrprob21}a). By 22 UTC to 23 UTC, a large area of low PROB1H-CREF25 values develops, centered on CI1 (Fig. \ref{sfc1hrprob22}a). This is also reflected in the MEAN-CREF25 field, with the earliest MEAN-CREF25 area occurring near 22:30 UTC (Fig. \ref{sfcmeancref}a). By 23 UTC, PROB1H-CREF25 magnitudes are largest near CI1 within the eastern half of the area where CREF25 was observed (Fig. \ref{sfc1hrprob23}a). The low values of PROB1H-CREF25 along the western edge of CI2, noted in the previous section (Fig. \ref{sfc3hcrefprob}a), develop within this hour. Thus, the ensemble members that do produce convection in this region initiate storms more than an hour later than observations.

In CI3, PROB1H-CREF25 values are (\textless 20\%) in the 21 UTC -- 22 UTC period, in good agreement with the locations of initial development (Fig. \ref{sfc1hrprob21}a). MEAN-CREF25 areas are only associated with the southern storm in CI3, where PROB1H-CREF25 values are higher (this demonstrates the disadvantage of using ensemble mean forecasts of convection for low probability events; Fig. \ref{sfcmeancref}a). In CI3, non-zero PROB1H-CREF25 values expand and increase in magnitude during the following hour, with the southern supercell having larger probabilities (Fig. \ref{sfc1hrprob22}a). Later, the PROB1H-CREF25 envelope broadens within CI3, with some members producing convection west of the observed CREF25 area during this period, resulting in low probabilities (Fig. \ref{sfc1hrprob23}a). The lower PROB1H-CREF25 values in CI3 compared to the probabilities in CI1 imply uncertainty with the development of convection in CI3, potentially both with CI, and for those members that do produce convection, with the exact placement of the storms (i.e. larger spread).

\subsection{Results from SFC1H, SFC2H, and SFC3H}
The largest differences between the three ensemble forecasts occur within CI2 and CI3. For CI3, improvements are evident in the ensemble mean, as the development of the northern storm within CI3 is captured in the MEAN-CREF25 field in SFC2H and SFC3H (Fig. \ref{sfcmeancref}c-d), while it is not in CNTL and SFC1H (Fig. \ref{sfcmeancref}a-b). Second, as the DA period is increased, a westward shift occurs in the placement of CI within CI3, in better agreement with observations. Finally, the timing of CI is much improved, especially in SFC3H, with MEAN-CREF25 areas occurring as early as 21:30 UTC for the northern storm (Fig. \ref{sfcmeancref}d), while CNTL, SFC1H, and SFC2H delay initiation to 22 UTC or later in this region (Fig. \ref{sfcmeancref}a-c). Less substantial changes in the forecasts of MEAN-CREF25 occur for CI2. The MEAN-CREF25 area shifts westward in CI1 as the DA period is increased, but the timing of CI is further delayed, well after the time of observed CI shortly after 21 UTC (Fig. \ref{sfcmeancref}d). Very small areas of MEAN-CREF25 occur within CI2, the size and timing of which are nearly unchanged between the three experiments. The spurious convective development in CI4 is also unchanged across the set of four experiments.

Differences between the CNTL and SFC3H forecasts of PROB3H-CREF25 provide more insight into the differences in the MEAN-CREF25 forecast between the experiments (Fig. \ref{sfc3hcrefprob}). In CI3, where the largest changes to MEAN-CREF25 occurred, PROB3H-CREF25 values are greater in SFC3H, especially associated with the northern storm (probabilities \textgreater 80\%). Here, an axis of higher probabilities (35-50\%) exist associated with ensemble members that produce and maintain a left-split that moves toward the northeast. This was also evident in CNTL, but the PROB3H-CREF25 values are more focused, and more members predict this forecast scenario. While very small differences existed in the MEAN-CREF25 field within CI2 between the four experiments, the PROB3H-CREF25 values are larger in SFC3H than CNTL. The areas of non-zero PROB3H-CREF25 values in SFC3H coincide with the western edge of CI2, with between 35-50\% of ensemble members indicating CI, including a right-moving supercell (not shown). The envelope of probabilities associated with CI1 is shifted westward and is more refined, matching well with the observed area of CREF25, although PROB3H-CREF25 magnitudes are reduced in SFC3H. In CI4, one area of CI was forecast in CNTL, but several additional areas are indicated in SFC3H, thus the surface DA appears to exacerbate the spurious convection within this region.

The forecast centroids suggest a large amount of uncertainty in the placement of convection within CI1 in both CNTL and SFC3H (Fig. \ref{sfccentroids}). A large spread of centroid points is present, although they envelope the observed centroid tracks. In SFC3H, some members produce convection earlier in the period (i.e. there are more blue centroid points in SFC3H than CNTL), and these are located further to the northwest in SFC3H. This gives a different impression of timing errors compared to the delayed CI suggested by the MEAN-CREF25 field. While PROB3H-CREF25 values were near zero in CNTL within CI2, the forecast centroid tracks show that several ensemble members produced convection at the end of the forecast period in this region, although the points are widely scattered and occur much later than observed. In SFC3H, a tightly clustered track of centroid points is forecast very close to the location and time of observed CI.

\begin{figure}
\centering
\includegraphics[scale=0.75]{surfaceDA_3HCREFPROB}
\caption{Three-hour (21 UTC to 00 UTC) forecast probabilities (shaded) of CREF \textgreater 25 dBZ for (a) CNTL and (b) SFC3H. Maximum observed CREF \textgreater 25 dBZ from 21 UTC to 00 UTC denoted by black contour.}
\label{sfc3hcrefprob}
\end{figure}
\begin{figure}
\centering
\includegraphics[scale=0.75]{surfaceDA_1HCREFPROB21}
\caption{Same as Fig. \ref{sfc3hcrefprob} except for one-hour (21 UTC to 22 UTC) probabilities and maximum observed CREF contour.}
\label{sfc1hrprob21}
\end{figure}
\begin{figure}
\centering
\includegraphics[scale=0.75]{surfaceDA_1HCREFPROB22}
\caption{Same as Fig. \ref{sfc3hcrefprob} except for one-hour (22 UTC to 23 UTC) probabilities and maximum observed CREF contour.}
\label{sfc1hrprob22}
\end{figure}
\begin{figure}
\centering
\includegraphics[scale=0.75]{surfaceDA_1HCREFPROB23}
\caption{Same as Fig. \ref{sfc3hcrefprob} except for one-hour (22 UTC to 23 UTC) probabilities and maximum observed CREF contour.}
\label{sfc1hrprob23}
\end{figure}
\begin{figure}
\centering
\includegraphics[scale=0.75]{surfaceDA_meancref}
\caption{Ensemble mean CREF areas \textgreater 25 dBZ color-coded at 15-minute intervals between 21 UTC and 00 UTC, with earlier areas plotted on top of later areas. Three-hourly maximum observed CREF \textgreater 25 dBZ areas plotted with black contour.}
\label{sfcmeancref}
\end{figure}
\begin{figure}
\centering
\includegraphics[scale=0.7]{surfaceDA_centroids}
\caption{Centroids of forecast CREF25 objects from each ensemble member (colored circles) and observed CREF25 objects (colored circles with black outlines) at 15-minute intervals between 21 UTC and 00 UTC (colored circles). The creation process of the centroids are detailed in section \ref{fcstcompare}. }
\label{sfccentroids}
\end{figure}

The centroids indicate a high degree of certainty in the placement and tracks of convection in CI3 in both CNTL and SFC3H (Fig. \ref{sfccentroids}). In SFC3H, the number of forecast centroid points associated with the northern storm has increased and their spread has been reduced. Further, a larger number of members produce a northeastward moving storm associated with a left-split. Differences in the position and timing of the forecast centroid points for the southern storm are less compared to the northern storm between the two experiments. Within CI4, the forecast centroid points also indicate that the spurious PROB3H-CREF areas in SFC3H are associated with the development of three areas of convection; in CNTL, most members only predict spurious convection in one location.

\subsection{Discussion} 
\label{sfcDA_diss}
Taken together, the MEAN-CREF25, PROB3H-CREF, and forecast centroid tracks demonstrate that surface DA improved the forecasts of CI occurrence and timing within CI2 and CI3, while providing less benefit within CI1 and CI4. The CNTL ensemble was clearly deficient in producing convection within CI2 and had delayed CI in CI3. A potential reason for the delayed CI within CI3 in CNTL is the westward bias in the placement of the surface dry line; CNTL does not mix the dry line far enough east during the afternoon hours. Other studies have found that convection-permitting models mix the dry line too far to the east (Clark et al. 2012). The present case is different in that the dryline is not associated with a synoptic-scale low-pressure system, thus its evolution is likely more sensitive to PBL estimates of vertical mixing. The dry line and moisture biases will be discussed in section \ref{tdsection}.

Another potential reason forecasts were most improved in CI2 and CI3 are due to observation availability. CI2 and CI3 are relatively well observed (primarily by the Oklahoma mesonet), while fewer surface observations are available outside of Oklahoma within CI1 and CI4. The trends in CI timing and placement as the DA period increases are encouraging, especially within CI2 and CI3. The PROB3H-CREF25 magnitudes increase and become more refined as the time of CI approaches. In a real-world setting, probability trends between subsequent ensemble forecasts can give forecasters confidence in a particular forecast outcome. The precise effect of DA in these experiments is difficult to extract, since the forecast lead-time is not consistent. Thus, the differences in the above ensemble forecasts are likely due to a combination of changes in initial condition accuracy during DA and reduced predictability error with shorter lead times, making it difficult to disentangle each source of error on the resulting forecast. Two additional experiments were conducted in an attempt to understand these sources of error.

\subsection{Impact of the length of the DA period}
In a subsequent set of experiments, surface DA was delayed by one and two hours (assimilation period of 19 -- 21 UTC and 20 -- 21 UTC), so the end of the assimilation period coincides with 21 UTC. A cleaner comparison of the effects of DA length can be made with these two experiments (SFC1H-19UTC and SFC2H-20UTC), combined with SFC3H.

One hour of surface DA in SFC1H-20UTC has a clear impact on the PROB3H-CREF25 magnitudes compared to CNTL (c.f. Fig \ref{sfc3hcrefprob}a, \ref{sfc3hdalength}a). One hour of DA adjusts the envelope of probabilities westward within CI1, but magnitudes are reduced from \textgreater 80\% to  \textless 50\%. Near and within CI2, two areas of \textgreater 50\% probabilities are produced, compared to near zero probabilities in CNTL. In CI3, PROB3H-CREF25 values increase by 40\% -- 50\% compared to CNTL . A swath of probabilities is present in CI3 associated with the members producing a left-split. After additional hours of surface DA (Fig. \ref{sfc3hdalength}b,c), PROB3H-CREF25 magnitudes increase within CI1, but are observed to decrease in other areas. In CI3 and CI4, PROB3H-CREF25 values remain similar to SFC1H-19UTC for the northern supercell within CI3, while PROB3H-CREF25 values decrease for the southern supercell. This is true for several areas of spurious convection within CI4 as well, although the probabilities increase associated with the spurious storm just to the south of the southern supercell in CI3.

\begin{figure}
\centering
\includegraphics[scale=0.6]{surfaceDA_dalength}
\caption{As in Fig. \ref{sfc3hcrefprob}, but for (a) SFC1H-20UTC, (b) SFC2H-19UTC, and (c) SFC3H.}
\label{sfc3hdalength}
\end{figure}

The additional two experiments provide a clearer interpretation of the effect of multiple hours of surface DA on short-term forecasts of convection. In CI2 and CI3 probability values within SFC1H-20UTC are similar to those from SFC3H, thus it appears that in these two areas, the primary source of forecast improvement was the one-hour of surface DA between 20 UTC to 21 UTC, immediately prior to CI (Fig. \ref{sfc3hdalength}a-c). Extending the DA period to two or three hours (as in SFC2H-19UTC and SFC3H) either has a small benefit, or is slightly detrimental to the PROB3H-CREF25 probabilities (e.g. reduction of magnitudes between SFC2H-19UTC and SFC3H). The longer period of DA is partially beneficial in CI3, since probability values increase between SFC1H-20UTC and SFC3H for the northern supercell, but decrease for the southern supercell. As noted before, a major role in these differences may be due to observational availability. In areas that are well observed (e.g. CI2 and CI3), the relatively high-density of observations, combined with frequent assimilation cycles, is able to adjust the state more quickly, thus reducing the need for a longer assimilation period. In more sparsely observed areas (e.g. CI1), the longer assimilation period is beneficial in order to reap improvements from better observed regions through forward integration of the model dynamics.

\subsection{CNTL and SFC3H comparison}
A comparison between the CNTL and SFC3H states is useful to understand the impacts and behavior of the surface DA process. Differences in the surface moisture field and in the structure of the lower troposphere will be highlighted in the following two subsections.

\subsubsection{Differences in surface moisture field}
\label{tdsection}
CI can be especially sensitive to the distribution of boundary layer moisture \citep{crook96,weckwerth00}. Specific humidity variations of 1 g kg\textsuperscript{-1} can impact the occurrence of CI. Differences in the ensemble mean dew point field between CNTL and SFC3H were analyzed to gauge the impact of surface DA on the distribution of surface moisture (Fig. \ref{sfcmeantd}). The CNTL ensemble mean dew point forecast at 21 UTC possesses an approximately \(4.5^{\circ}C\) moist bias (Fig. \ref{sfcmeantd}a), with the largest errors within the domain located in two regions. The first is associated with the surface dry line, where observations indicate that the placement of the dryline in the forecast is too far west. The second is an axis that stretches from southcentral KS into southeast OK, near and to the northeast of a weak stationary boundary. The three hours of surface DA in SFC3H reduces both of these surface moisture biases. The largest differences between CNTL and SFC3H are in these two regions (Fig. \ref{sfcmeantddiff})

\begin{figure}
\centering
\includegraphics[scale=0.8]{surfaceDA_ensmeanTd}
\caption{Ensemble mean 2-m dew point temperature (shaded; degrees Fahrenheit) and 10-m wind field at 21 UTC 29 May 2012 from (a) CNTL and (b) SFC3H. Filled circles are differences between the ensemble mean dew point analysis and observations (forecast minus observed). }
\label{sfcmeantd}
\end{figure}
\begin{figure}
\centering
\includegraphics[scale=0.8]{surfaceDA_ensmeanTd_diff}
\caption{Difference field (SFC3H-CNTL) for the ensemble mean 2-m dew point temperature (shaded; degrees Fahrenheit) and 10-m wind field at 21 UTC 29 May 2012. Full wind barbs indicate a wind difference of 5 m s\textsuperscript{-1}, while half barbs is a difference of 2.5 m s\textsuperscript{-1}. }
\label{sfcmeantddiff}
\end{figure}

The improvements in CI in these areas between CNTL and SFC3H are likely driven by these differences in the surface moisture field and the simulated dry line circulation. Not only is the dry line placement improved in SFC3H, but the magnitude of the differences between the theta-e of the two airmasses is also larger, due to lower dew points behind the dryline across SW OK and NW OK and little to no change in the surface dew point ahead of the dryline in western OK. The increased theta-e difference in SFC3H is associated with a stronger dryline circulation and surface mass convergence in the areas where convection was observed to initiate. This is further reflected in the ensemble mean surface wind field differences between the two experiments, with the three hours of surface DA resulting in increased in westerly momentum behind the dryline and easterly momentum ahead of the dry line.

\subsubsection{Differences in vertical profiles}
Surface DA has the ability to also make adjustments to the state above the surface. To illustrate the changes in vertical profiles between CNTL and SFC3H, vertical temperature and dew point profiles from each of the 50 posterior ensemble members at 23 UTC at the KOUN (Fig. \ref{sfcda_oun}) and KAMA (Fig. \ref{sfcda_ama}) sounding locations were plotted on a skew-T diagram. These two stations are representative of the air mass ahead of and behind the surface dry line. The observed KOUN and KAMA radiosondes, launched at 23 UTC, are overlaid for comparison. The vertical profiles are only shown below 400 hPa; differences between the two experiments are small above this level, since surface observations can only modify the state below 8 km AGL due to vertical localization.
\begin{figure}
\centering
\includegraphics[scale=0.85]{surfaceDA_OUN}
\caption{Skew-T temperature (degrees Celsius) and dew point (degrees Celsius) profiles from ensemble members (gray) and ensemble mean (black) at 23 UTC 29 May 2012 from (a) CNTL and (b) SFC3H at KOUN. Observed temperature (red) and dew point (green) profiles from the 00 UTC 30 May 2012 KOUN radiosonde (launched at 23 UTC).}
\label{sfcda_oun}
\end{figure}
\begin{figure}
\centering
\includegraphics[scale=0.85]{surfaceDA_AMA}
\caption{Same as Fig.\ref{sfcda_oun}, but for the ensemble forecast and observed profile at KAMA.}
\label{sfcda_ama}
\end{figure}

The 23 UTC CNTL ensemble mean temperature profile agrees well with the KOUN observed sounding, with the ensemble showing very little spread throughout the profile. The largest temperature spread is in association with the height of the PBL. The PBL height in the ensemble mean is less than observed, with the sounding falling within the spread of the ensemble members. All of the members are too moist within the boundary layer and the observed profile falls well outside the envelope of the CNTL members. This is consistent with the moisture biases discussed in the previous section. Above the PBL, slightly better correspondence exists between the observed and CNTL ensemble mean within the inversion layer (850 –- 750 hPa).

Surface DA is unable to make significant changes to the temperature profile due to little ensemble spread in temperature (i.e. the ensemble gives little weight to observations when the ensemble spread is small relative to the observation error). The SFC3H mean moisture profile is approximately 0.5 g kg\textsuperscript{-1} less than the CNTL mean profile within the PBL, closer to the observed sounding. Yet, the boundary layer depth has been reduced in SFC3H compared to CNTL (neither experiment captured the inversion at the PBL top). Similar adjustments to the moisture profiles occur at KAMA. The moisture amount has been reduced within the PBL, although the height of the PBL has not deepened considerably, and has been reduced in many of the ensemble members.
\begin{figure}
\centering
\includegraphics[scale=0.85]{CNTL_hodo}
\caption{Hodographs from CNTL ensemble members (gray) at 23 UTC 29 May 2012 and observed 00 UTC 30 May 2012 hodograph (black) at KOUN. Points on each hodograph indicate the wind speed/direction (kts) at 2 km, 4 km, 6 km, 8 km, 10 km, and 12 km for each ensemble member (colored circles), the ensemble mean (colored circle with black outline), and the observed hodograph (colored stars).}
\label{hodo}
\end{figure}

The 23 UTC wind profile in both CNTL and SFC3H is 10-15 knots slower than was observed at KOUN below 3 km AGL (Fig. \ref{hodo}). It appears that surface DA is not able to correct this lower-tropospheric wind bias since CNTL and SFC3H have similar errors. Discussion about the effects of these errors on predicted supercell storm motion is provided in Part III, since this bias is expected to have the largest impact later in the forecast period as convection matures.

\subsection{Impact of mesonet data and frequent cycling}
Two additional assimilation experiments were conducted to assess the impacts of both the mesonet dataset and the 5-minute assimilation interval on the analyses. In the first experiment, mesonet data were withheld; the remaining METAR observations were assimilated at 5-minute intervals from 18 UTC -- 21 UTC (SFC3H-NOMESO). In a second experiment, observations were assimilated once per hour (SFC3H-HOURLY). In SFC3H-HOURLY, METAR observations taken between 15-minutes before and 15-minutes after the top of the hour were assimilated, along with mesonet observations within 2.5 minutes of the top of the hour. The smaller mesonet assimilation window in SFC3H-HOURLY was chosen to ensure that each mesonet observing site only contributed, at most, one observation per assimilation cycle. SFC3H-HOURLY was designed to mimic hourly, cycled, DA systems that have been used in previous studies (e.g. \citealt{wheatleyetal12}).

\begin{figure}
\centering
\includegraphics[scale=0.7]{surfaceDA_ensmeanTd_diff3exp}
\caption{As in Fig. \ref{sfcmeantddiff}, but for (a) SFC3H-HOURLY minus CNTL, (b) SFC3H-NOMESO minus CNTL, and (c) SFC3H minus CNTL.}
\label{sfcmeantddiff3exp}
\end{figure}

The 21 UTC ensemble mean surface dew point and wind fields in SFC3H-NOMESO and SFC3H-HOURLY share more similarities to CNTL than to those from SFC3H (Fig. \ref{sfcmeantddiff3exp}a-b). The largest adjustments in SFC3H-HOURLY are made in association with the dry line in SW OK and NW TX, extending into west-central TX (Fig. \ref{sfcmeantddiff3exp}a). Even smaller differences are present between SFC3H-NOMESO and CNTL (Fig. \ref{sfcmeantddiff3exp}b). Given that neither experiment is capable of reducing the moisture biases nor fully adjusting the position and magnitude of the dry line, it appears the combination of frequent assimilation cycles and higher-resolution mesonet data that provides the largest impact on the analyses.

To determine if the adjustments made by the surface DA persist into the free forecast period, the ensemble mean dew point RMSE was computed every 5-minutes using METAR and mesonet observations for the four experiments (CNTL, SFC3H, SFC3H-NOMESO, SFC3H-HOURLY; Fig. \ref{sfcmeantdrmse}). The initial differences in error between the experiments at the initial analysis time continue throughout the forecast. Following a period of fairly rapid error growth during the first 30 minutes, the RMSE for all experiments stabilize. The SFC3H ensemble mean has the lowest RMSE throughout the 3-hour period. CNTL and SFC3H-NOMESO are largely indistinguishable during the entire period, except for the first 15-minutes when SFC3H-NOMESO has a slightly smaller RMSE than CNTL. The RMSE for the SFC3H-HOURLY falls approximately midway between SFC3H and CNTL during the period. 
\begin{figure}
\centering
\includegraphics[scale=0.6]{surfaceDA_rmseTd}
\caption{Root-mean squared dew point temperature error (K) for the 21 UTC to 00 UTC ensemble mean forecast from various surface DA experiments.}
\label{sfcmeantdrmse}
\end{figure}
Assimilating METAR and mesonet observations once per hour is able to cut the dew point errors in half in this case, with an equally large amount of RMSE reduction if the observations are assimilated every 5-minutes. In the latter case, the benefits are primarily due to the ability to utilize the mesonet data that occur beyond the restrictive mesonet data window used herein. It is uncertain what effect broadening the mesonet assimilation window would have on the analyses, to include observations further away from the analysis time. Doing so would frequently result in multiple observations, valid at the same location but at different times, being assimilated in an analysis valid at a single time. This would likely have the largest effect where observations are rapidly changing, for example in the vicinity of fronts and, in the present case, a surface dry line. 

\begin{figure}
\centering
\includegraphics[scale=0.7]{surfaceDA_3HCREFPROB_diff3exp}
\caption{As in Fig. \ref{sfc3hcrefprob}, but for (a) SFC3H-HOURLY, (b) SFC3H-NOMESO, and (c) SFC3H.}
\label{3hcrefprob_diff3exp}
\end{figure}

PROB3H-CREF25 values for the two experiments generally fall in between CTRL and SFC3H (Fig. \ref{3hcrefprob_diff3exp}). One notable exception is the probabilities associated with CI2; in this area both SFC3H-HOURLY or SFC3H-NOMESO are similar to the CTRL in that they produce little to no convection. Only SFC3H has ensemble members that develop sustained convection within CI2.

Assimilating surface observations as frequently as every 5-minutes could lead to imbalance in the resulting analyses, potentially to the detriment of forecast accuracy \citep{greybushetal11}. The surface pressure tendency (PSFCTEND) provides a way to measure the degree of imbalance in the initial analyses. Thus, the ensemble mean domain maximum PSFCTEND (that is, the average of the 50 values of domain maximum PSFCTEND) was computed at 5-minute intervals in through the forecast period for the ensemble forecasts from the three assimilation experiments (SFC3H, SFC3H-HOURLY, and SFC3H-NOMESO). Only the absolute magnitude of the PSFCTEND was used in the computation.

\begin{figure}
\centering
\includegraphics[scale=0.6]{surfaceDA_presstend}
\caption{Ensemble-averaged, domain-averaged, absolute surface pressure tendency (Pa sec\textsuperscript{-1}) computed at 5-minute intervals for a 6-hr forecast beginning at 21 UTC from SFC3H (blue), SFC3H-HOURLY (black), SFC3H-NOMESO (green), and CNTL (red).}
\label{sfcdapresstend}
\end{figure}

The peak in surface pressure tendency is similar across the three experiments, with domain average PSFCTEND between 0.75 and 0.85 Pa s\textsuperscript{-1} (Fig. \ref{sfcdapresstend}). For SFC3H, PSFCTEND decreases more gradually than SFC3H-HOURLY and SFC3H-NOMESO. By 2 hours into the forecast, the experiments have indistinguishable PSFCTEND values. Thus, imbalance is slightly larger in SFC3H and takes longer to stabilize than the other two experiments, even though SFC3H-NOMESO also uses a 5-minute assimilation frequency (although with a much smaller number of assimilated observations).

\subsection{Sensitivity to horizontal localization cutoff}
As described in Part I, the choice of an appropriate length scale for localization is a complex function of observation type, density, state variable, etc. Some results are presented here to illustrate differences in the surface dew point and forecasts of convection when larger horizontal localization values are used. The inclusion of mesonet data in the current study results in a varying observation density across the domain (Fig. \ref{surfobsdistribution}). The horizontal localization cutoff in all of the experiments so far (60 km) was chosen to be approximately double the distance of the average observation spacing within Oklahoma, the region of relatively high-density observations due to the presence of the Oklahoma mesonet. More sparsely observed regions might benefit from a larger horizontal cutoff. To test this hypothesis, SFC3H was repeated using a horizontal localization cutoff increased to 120 km (SFC3H-H120V8) and 240 km (SFC3H-H240V8).

\begin{figure}
\centering
\includegraphics[scale=0.8]{surfaceDA_localization}
\caption{As in Fig. \ref{sfc3hcrefprob}, but for (a) SFC3H, (b) SFC3H-H120V8, and (c) SFC3H-H240V8.}
\label{sfcdaloc}
\end{figure}

The differences in the ensemble mean dew point RMSE between the three experiments at the final analysis time (21 UTC) is relatively modest (approximately \(0.3^{\circ}C\)). Compared to SFC3H, the RMSE is smaller in SFC3H-H120V8, while SFC3H-H240V8 has a slightly larger RMSE. Larger differences emerge later in the forecast period, primarily after 21:30 UTC. Both SFC3H-H120V8 and SFC3H-H240V8 have smaller ensemble mean dew point RMSE than SFC3H, with SFC3H-H120V8 the smallest of the three experiments. Increasing the localization length results in a smoother analysis, which is likely the reason for the lower RMSE in SFC3H-H120V8, but increases to the localization beyond this length-scale result in slight RMSE increases in SFC3H-H240V8, although the RMSE remains less than SFC3H for the forecast period beyond 21:30 UTC.

The forecast RMSE implies that the SFC3H-H120V8 and SFC3H-H240V8 analyses both have better representations of the surface dew point field, yet these benefits do not necessarily extend into better short-term forecasts of convection. In fact, fairly large differences exist between the three experiments. In general, increasing the horizontal localization cutoff reduces probabilities of convection throughout the domain in the 21 UTC -- 00 UTC period (Fig. \ref{sfcdaloc}). One exception to this is in CI1, where PROB3H-CREF25 magnitudes are increased. In CI2, forecast probabilities are reduced to near zero in SFC3H-H240V8, while in CI3 a similar reduction occurs, although \textless 35\% of the ensemble continues to produce two areas of convection, albeit with much larger placement errors than in SFC3H or SFC3H-H120V8. In CI4, the reduction in forecast probabilities is beneficial since no convection is observed in this region. All three experiments continue to produce large values of PROB3H-CREF25 east of CI4 near KDFW; forecast probabilities are less sensitive here to localization than in other areas of the domain.

Two factors are likely influencing the differences in forecast probabilities between these three experiments: 1) spread reduction and 2) smoother analyses, both due to larger localization. As localization increases, the ensemble spread is reduced as state points are influenced by more observations (this effect was described in Part I). Over time, this spread collapse could result in observations having little weight on the analyses (i.e. filter divergence) as the ensemble becomes increasingly confident on a given forecast outcome. Unlike the experiments in Part I, the use of prior inflation in these experiments is designed to counteract this tendency, by increasing the prior ensemble spread. If prior state inflation were insufficient to ameliorate the tendency for spread collapse, probabilities of convection would likely be drawn to lower or higher values, as members cluster around solutions where convection develops or is suppressed. This behavior is observed between the three experiments, with CI1 CREF25 probabilities increasing (almost all members produce convection in this location), and probabilities decreasing in areas along the dry line in OK and TX.

Larger localization also tends to produce smoother analyses as state points are influenced by more observations. In the present experiments, this effect may reduce the magnitude of circulations along the dry line, the primary initiating boundary within the domain. Probabilities are indeed reduced along the dry line within OK and TX, although the increase in probabilities in SW KS cannot be explained by this effect. Given the reduced observation density within SW KS compared to nearby areas in OK, the larger localization may be improving the analyses in this region by incorporating information from far away observations, a potentially more substantial effect than that of reducing the circulations associated with surface boundaries. If this were occurring, a similar effect would expect to be seen in other areas of limited observations, e.g. along the dry line in central TX; this is not observed to occur in this region.

\subsection{Summary and Discussion}
Results from several surface DA experiments were discussed in the present chapter. In the primary experiment, SFC3H, surface data, including METAR and mesonet data, were assimilated every 5-minutes during a three-hour period extending from 18 UTC and ending at 21 UTC, shortly before CI. High-frequency assimilation of surface data led to improvements in forecasts of CI, both in the timing and placement of initial convective development. These improvements extended into the short-term forecasting period due to a more accurate representation of the surface moisture field, and reduction of errors due to model biases. Additional experiments that only assimilated routine surface observations (i.e. METAR), or only assimilated observations each hour, did not see the same improvements in the forecasts of CI. Finally, increases to the horizontal localization cutoff for the assimilated surface observations led to detrimental forecasts of CI for this case.

Frequent assimilation of surface observations, including routinely issued METARs and observations from mesoscale networks, improved forecasts of CI and subsequent convective evolution where the development of convection is driven largely by surface boundaries (e.g. a dry line). This is the first time that high-frequency mesoscale surface data have been assimilated, with a rapid cycling period using the EnKF, for the explicit goal of predicting CI for a real convective event. Given the challenges associated with forecasting CI, this is an encouraging result, yet several challenges remain that were also documented within this section. Those involve model error from PBL parameterizations and the effects that surface observations have on the overlying free atmosphere.

The MYJ PBL parameterization scheme used within these experiments produced forecasts with a positive moisture bias, as noted in previous evaluations \citep{huetal10}. A similarly configured mesoscale ensemble system to the one used herein also contained a positive moisture bias within the PBL that impacted forecasts of convection \citep{romineetal13}. In cycled DA systems, like the system that provided the initial and boundary conditions for the present experiments, these model errors can persist and grow with each analysis cycle, unless observations are regularly assimilated to constrain the model solution. It appears the 5-min cycling frequency used herein is able to reduce the moisture bias, in addition to correcting the placement and strength of surface boundaries and their associated surface circulations. Even though surface DA improves the analyses, assimilating surface observations with biased PBL parameterizations may still be suboptimal \citep{deedasilva98}. Future work should investigate the use of techniques to reduce the effects of model errors, particularly PBL parameterization error, on meso- and convective-scale EnKF analyses.

Appropriately adjusting the state above the surface during surface DA is an additional challenge to the successful assimilation of real surface observations. While surface fields may be improved, the state above the surface may not necessarily be more accurate. In the current experiments, model error is again a likely culprit. While a suboptimal update due to the surface moisture biases likely plays a role in these errors, examination of the soundings suggests that errors within the PBL parameterization that describe mixing, entrainment, and PBL growth may have led to erroneous covariances between observations and the model state. For example, at both KOUN and KAMA, the boundary layer was shallower and moister than the observed profile in the CNTL ensemble without surface DA. Thus, after a period of surface DA, it would be encouraging to see a deeper PBL, along with a corresponding reduction in mixing ratio. Although the SFC3H mean PBL mixing ratio was indeed less, the PBL depth became shallower, instead of the desired deepening. Further research is crucial to fully understand the role of model error due to PBL parameterization and to begin to develop strategies to properly handle surface observations and their impact above the surface. One simple strategy is to reduce the vertical localization length-scale, thus removing the impact of surface observations above a specified height.

The improvements to forecasts of CI and evolution in most of the domain by assimilating surface observations is taken to be a sign that the mesoscale environment is more accurately represented in the final ensemble analyses after multiple hours of surface DA. These benefits should extend to analyses and forecasts that are derived from experiments where radar data are assimilated once they become available following CI. In Part II, the impact of radar DA compared to surface DA for this case will be investigated.
