\chapter{Motivation}
Initializing numerical models with convective-scale information is essential for the accurate prediction of thunderstorms \citep{lilly90}. Yet, several obstacles must be overcome before such a goal is achieved. First, routinely available observations that provide information on in-storm convective-scale structures are limited to Doppler radial velocity and reflectivity, yet these are indirectly, and non-linearly, related to the set of state variables needed for model initialization. This presents challenges when using DA techniques such as the ensemble Kalman filter (EnKF), which has been actively employed to generate reliable convective-scale initial conditions. Nevertheless, the EnKF has proved itself as a capable and natural method for this objective, given its use of an ensemble of states to derive the forecast error statistics needed for DA \citep{snyderzhang03}. Following assimilation, this ensemble can be exploited to initialize a convective-scale ensemble forecast. Yet, few studies have utilized forecasts initialized with EnKF assimilation systems; more effort has been expended assessing the EnKF-generated storm-scale analyses. Further, the studies that have examined forecasts have been restricted to isolated convective events over small domains (e.g. supercells) and less on larger-scale convective scenarios (e.g. squall lines, storm mergers, etc.). 

Unique challenges exist with these more complex events that have been not been adequately addressed. While predicting the evolution of deep convection is inherently a multiscale problem, most studies have initialized convection within horizontally homogeneous ICs. More complex events over larger domains require an accurate depiction of the mesoscale environment to capture time-varying, spatially heterogeneous features important in driving convective evolution (e.g. surface boundaries). Further, successfully predicting the characteristics of the near-surface environment is crucial to capture CI and later convective evolution. Observations from surface networks, especially those networks that contain mesoscale information, can ensure that the pre-convective environment supports convection, and also in the post-CI environment as convective outflow modifies the boundary layer. Thus, assimilating both surface and radar observations will likely lead to larger gains in forecast accuracy than assimilating radar or surface data alone for complex convective events. Assimilating surface observations with the EnKF before CI, in anticipation of convection, 

This dissertation will address several of these open questions. In Part I, observing system simulation experiments of a developing convective system were performed to test, in a controlled manner, the effects that certain EnKF assimilation strategies have on the analyses and forecasts of a system that grows upscale through cell and cold pool mergers. As a complement to these idealized experiments, a real-data case study was performed to look at the effects of various data sources on forecasts of the 29 May 2012 convective event that contained several similar characteristics of the OSSEs (e.g. cell mergers, upscale growth, splitting storms). In Part II, the effects of surface DA on the pre-convective mesoscale environment of this event are analyzed, with emphasis on forecasts of convection initiation. Finally, radar data, in addition to surface DA, is performed and short-term forecasts are produced and verified to gauge the mutual contributions of radar and surface data.
