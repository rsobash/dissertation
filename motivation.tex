\chapter{Motivation}
Initializing numerical models with convective-scale information is essential for the accurate prediction of thunderstorms \citep{lilly90}. The ensemble Kalman filter (EnKF) is one such technique that has proved itself as a capable and natural method for convective-scale data assimilation (DA), given its use of an ensemble of states to derive the needed forecast error statistics \citep{evensen94,snyderzhang03}. Unfortunately, routinely available observations that provide information on in-storm convective-scale structures are limited to Doppler radial velocity and reflectivity, yet these are indirectly, and non-linearly, related to the set of state variables needed for model initialization. Non-linearities violate several key assumptions underlying DA techniques \citep{kalnaybook}, but nevertheless, radar data has  been used with success to generate convective-scale initial conditions \citep{sun05}.

While past studies have assessed the accuracy of short-term forecasts of convective storms initialized with Doppler radar observations using an EnKF system, these cases have primarily been isolated convective events within small domains (e.g. supercells; \citealt{dowellwicker09,dowelletal11,dawsonetal12}, among others). The use of small domains and short-forecast lengths has led to the majority of convective-scale DA studies initializing convection within a horizontally homogeneous environment (a thorough discussion of this practice is provided in \citealt{dawsonetal12}). Forecasts of larger-scale convective systems (e.g. mesoscale convective systems; \citealt{wheatleystensrud10,snooketal12}) or using more regional domains containing a realistic variety of convective structures have received less attention. However, since the evolution of deep convection is affected by interactions between the mesoscale and convective-scale, a better depiction of the time-varying, spatially heterogeneous mesoscale environment should improve storm-scale forecasts \citep{stensrudgao10,stensrudetal13}. 

The sensitivity of the combined mesoscale and storm-scale analyses and forecasts to observational data frequency and various components of the DA system (e.g. localization) need to be systematically examined for cases with a variety of convective structures and modes embedded within a complex mesoscale environment. Forecasts of these types of complex events will also need to predict convection initiation (CI) accurately, which represents a significant challenge \citep{kainetal10}.

Building on the successes and limitations of previous work, this dissertation explores a variety of issues that could lead to further gains in convective-scale forecast skill using an EnKF DA system, with a particular emphasis on forecasts of more complex convective events (e.g. events composed of cell mergers, upscale growth, splitting storms, etc.). These issues include: 1) the choice of covariance localization for radar data from the WSR-88D network, 2) the impact of sub-hourly surface mesonet DA on mesoscale analysis quality and CI forecast skill, 3) the relative contributions of assimilating surface mesonet vs. WSR-88D data on convective-scale forecast skill, and 4) model and forward operator errors that reduce the effectiveness of assimilating surface and radar datasets. Studies assimilating surface mesonet data and radar data together have been limited (e.g. \citealt{schenkmanetal11b}). Keeping in mind the shortcomings of each approach, both observing system simulation experiments (OSSEs) and real-data experiments are used as tools to gain a more complete understanding of the issues noted above.
