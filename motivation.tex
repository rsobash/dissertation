\chapter{Motivation}
Initializing numerical models with convective-scale information is essential for the accurate prediction of thunderstorms \citep{lilly90}. Yet, several obstacles must be overcome before such a goal is achieved. First, routinely available observations that provide information on in-storm convective-scale structures are limited to Doppler radial velocity and reflectivity, yet these are indirectly, and non-linearly, related to the set of state variables needed for model initialization. Non-linearities violate several key assumptions underlying data assimilation (DA) techniques \citep{kalnaybook}, but nevertheless, radar data has  been used with success to generate reliable convective-scale initial conditions \citep{sun05}.

The ensemble Kalman filter (EnKF) is one such technique that has proved itself as a capable and natural method for this objective, given its use of an ensemble of states to derive the forecast error statistics needed for DA \citep{evensen94,snyderzhang03}. Following assimilation, this ensemble can be exploited to initialize a convective-scale ensemble forecast. Some studies have assessed the accuracy of short-term forecasts of convective storms initialized with real data and an EnKF system, yet these cases have primarily been isolated convective events within small domains (e.g. supercells; \citealt{dowellwicker09,dowelletal11,dawsonetal12}, among others). Forecasts of larger-scale convective systems (e.g. mesoscale convective systems; \citealt{wheatleystensrud10,snooketal12}) or using more regional domains containing a realistic variety of convective structures have received less attention. Further, the sensitivity of the storm-scale analyses and forecasts to various components of the DA system (e.g. localization) have not been systematically examined.

The evolution of deep convection is affected by interactions between the mesoscale and convective-scale, thus a better depiction of the time-varying, spatially heterogeneous mesoscale environment should improve storm-scale forecasts \citep{stensrudetal13}. Yet, many studies have initialized convection within a horizontally homogeneous environment (a thorough discussion of this practice is provided in \citealt{dawsonetal12}). Before CI, assimilation of sub-hourly observations from mesoscale surface data networks (i.e. mesonets) should improve forecasts of CI and ensure the mesoscale environment is properly represented once radar data become available, more so than traditional surface data sources. Following CI, these surface data can continue to correct mesoscale errors, in addition to errors associated with convectively-generated features at the surface, such as cold pools \citep{wheatleystensrud10}. The use of sub-hourly mesoscale surface data to improve forecasts of CI has not been previously documented. In addition, studies assimilating surface mesonet data and radar data together have been limited (e.g. \citealt{schenkmanetal11b}).

Building on the successes and limitations of this previous work, this dissertation explores a variety of issues that could lead to further gains in convective-scale forecast skill using an EnKF DA system, with a particular emphasis on forecasts of more complex convective events (e.g. events composed of cell mergers, upscale growth, splitting storms, etc.). These issues include: 1) the choice of covariance localization for radar data from the WSR-88D network, 2) the impact of sub-hourly surface mesonet DA on mesoscale analysis quality and CI forecast skill, 3) the relative contributions of assimilating surface mesonet vs. WSR-88D data on convective-scale forecast skill, and 4) model and forward operator errors that reduce the effectiveness of assimilating surface and radar datasets. Keeping in mind the shortcomings of each approach, both observing system simulation experiments (OSSEs) and real-data experiments are used as tools to gain a more complete understanding of the issues noted above.
